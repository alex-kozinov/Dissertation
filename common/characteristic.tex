%% Рассматриваемый в дипломной работе робот состоят из неподвижных частей и сервомоторов, которые обеспечивают вращение вдоль определённой оси в заданном диапазоне градусов. 

Задача прямохождения двуногих роботов уникальна не только тем, что позволяет перемещаться по рельефной поверхности и преодолевать препятствия, но ещё и тем, что качество перемещения зависит от множества факторов окружающей среды и конструкции робота. Движение обеспечивается изменением положения подвижных деталей с заданной частотой. Алгоритм ходьбы генерирует последовательность движения сервоприводов, которая может быть постоянной, а может изменяться в ходе исполнения алгоритма, реагируя на меняющиеся условия.

Важно уметь решать эту задачу для произвольного робота. Так как в отличии от коммерческих моделей, которые поставляются с готовым функционалом, самодельные роботы не поддерживают доступные алгоритмы ввиду особенностей отдельных деталей.

Одним из первых решений является подход полного расчёта последовательности углов сервомоторов, основанный на физической модели робота, а также представлении походки в виде циклическом изменении положения стоп робота [TODO: ссылка]. Чтобы исправить проблему с низкой адаптивностью, в физическую модель была добавлена точка центра масс [TODO: ссылка]. Были также подходы, в которых в качестве основной модели движения выбирается походка человека, а не спроектированная последовательность [TODO: ссылка]. Но все эти подходы обладают одним большим недостатком: они опираются на расчёты для заданной физической модели робота и предполагают, что ходьба совершается по ровной поверхности.

Вместе с математическим подходом развивается альтернативный подход, использующий методы машинного обучения. Один из первых алгоритмов опирался на конструкцию робота и оптимизировал последовательность стоп [TOD: ссылка] %% Stochastic policy gradient reinforcement learning on a simple 3D biped
. Этот подход позволял находить более оптимальные паттерны движения, но проблемы математического подхода оставались открытыми. В [TODO: ссылка] %% Machine Learning Algorithms in Bipedal Robot Control
были представлены подходы к обучению агента, где в качестве пространства состояний были выбраны углы поворотов моторов. Что довало большую свободу в выборе подходящей стратегии движения. Но алгоритм всё равно ограничен, так как использовал дискретизированную версию действий агента – допускались повороты углов лишь на заданной конечное множество углов

В текущей работе будет изучена возможность применения методов обучения с подкреплением для задачи прямохождения самодельного робота. Задача будет решаться в симуляторе с упрощённой моделью, которая сохранила физические особенности реального робота. Итогом работы является агент, который в качестве состояния принимает два непрерывных вектора~--- вектор положений моторов и вектор ориентации робота относительно поверхности