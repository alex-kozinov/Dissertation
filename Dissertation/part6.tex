\chapter{Предложения по содержанию дальнейшей работы}\label{ch:ch6}

Вся запланированная работа направлена на возможность применения обученной в симуляторе модели на реальном роботе. Должны быть решены следующие проблемы: низкая производительность процессора на роботе, несоответсвие свойств реальной среды свойствам симуляции или неточная их репрезентация в симуляции.

Проблему с производительностью можно разбить на несколько задач. Можно упростить размерность пространства состояний и изучить возможность обучения корректных моделей ходьбы. Так как основные вычислительные затраты приходятся на применение нейронной модели агента, то можно исследовать более простые модели, позволяющие производить более быстрое выполнение на слабом процессоре. Если предыдущие подходы не помогут, то можно использовать обученную модель как дополнение к базовой модели, и использовать её в случае, когда робот начинает терять равновесие и базовая модель не сможет справиться. 

Для того, чтобы обученная модель робота без ошибок воспроизводила ходьбу из симуляции, нужно, чтобы во время обучения среда в симуляции была максимально приближена к реальному миру. Поэтому можно попробовать добавить прилагаемые в случайные моменты времени силовые воздействия к роботу, которые могут выводить его из равновесий. Эти действия будут симулировать, например, столкновения с другими роботами или со статичными предметами. Также можно добавить небольшие неровности на полу или рассыпать подвижные маленькие сферы, которые будут имитировать сопротивление коврового покрытия. Такие эксперименты скорее всего ограничат скорость робота, но полученная ходьба будет готова к внезапным изменениям в состоянии робота. 