%% Рассматриваемый в дипломной работе робот состоят из неподвижных частей и сервомоторов, которые обеспечивают вращение вдоль определённой оси в заданном диапазоне градусов. 

Задача прямохождения двуногих роботов уникальна не только тем, что позволяет перемещаться по рельефной поверхности и преодолевать препятствия, но ещё и тем, что качество перемещения зависит от множества факторов окружающей среды и конструкции робота. Движение обеспечивается изменением положения подвижных деталей с заданной частотой. Алгоритм ходьбы генерирует последовательность движения сервоприводов, которая может быть постоянной, а может изменяться в ходе исполнения алгоритма, реагируя на меняющиеся условия.

Важно уметь решать эту задачу для произвольного робота. Так как в отличии от коммерческих моделей, которые поставляются с готовым функционалом, самодельные роботы не поддерживают доступные алгоритмы ввиду особенностей отдельных деталей.

Одним из первых решением является подход полного расчёта последовательности углов сервомоторов, основанный на физической модели робота, а также представлении походки в виде циклическом изменении положения стоп робота [TODO: ссылка]. У 