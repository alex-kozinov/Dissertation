\chapter{Обзор литературы}\label{ch:ch2}
В последних работах на похожие темы можно встретить несколько разных подходов не только к обучению агента, но и к самому формату управляющей программы. Так как в данной работе изучается влияние внешней среды симуляции и функции награды на результирующую походку, то важно сделать фокус на статьях, которые предлагают разные подходы по оцениванию качества походки модели.

В статье \cite{kumar2018bipedal} постановка задачи наиболее близка к постановке в текущей работе. Функция наград поощряет продвижение робота вперёд и штрафует за падения. Пространство состояний и действий имеют похожую структуру, но у модели робота отсутствуюут стопы и всего 6 степеней свободы, что является более простой моделью по сравнению с рассматриваемым нами. В статье \cite{8733632} изучается возможность обучения робота Nao прямохождению. Этот робот имеет более сложную физическую модель, но он также является гуманоидом. Функция награды имела вид $V_x - 0.2|V_y|-0.001[is\_fallen]$~--- таким образом награда стимулировала робота двигаться по направлению оси $OX$, что является целевым направлением, а также держать равновесие, так как за падение полагался штраф. В статье \cite{1573573} испльзовался двуногий робот, который был специально сконструирован для изучения ходьбы. В качестве награды использовалась сумма $200 footstep_{dx} - 8.3 motor_{dJ}$, где $footestep_{dx}$ обозначает расстояние, которое преодолела нога, совершавшая шаг, а $motor_{dJ}$~--- изменение в прилагаемом усилии в сервомоторах для совершения вращения. В статье \cite{liang2018gpuaccelerated} в качестве функции награды используется формула $2[not\_fallen]+V_x+0.5\cos{\Theta_z} + 0.5[\cos{\Theta_y > 0.93}] - 4\frac{||J||_1}{J_{\max}}$. В этой формуле первый раз появляются штрафы за отклонения робота от целевого направления~--- $\Theta_z$, а также робот награждается за прямую стойку, за которую отвечает угол поворота вдоль оси $OY$~---$\Theta_y$. Также в этой формуле есть штраф за высокое напряжение в моторчиках, только отличие от предыдущего подхода в том, что ранее штраф накладывался на большие изменения в прилагаемой силе, а здесь штрафуются высокие абсолютные значения в каждый момент симуляции. Тем не менее все описанные подходы имеют один существенный недостаток~--- веса перед различными параметрами являются абсолютными. Это значит, что для нового робота нужно будет самостоятельно подбирать эти параметры. Но так как проверка параметров на корректность требует много времени для обучения модели, то данные подходы хоть и демонстрируют основные необходимые элементы для определения качества ходьбы, но не являются универсальным подходом. В данной работе эти подходы будут использованы для получение более универсальной формулы, которую можно будет переиспользовать на роботах с другими физическими свойствами.
