\chapter{Выводы}\label{ch:ch5}
Достигнутые результаты демонстрируют возможность осуществления довольно быстрой процедуры обучения робота ходьбе. Эта скорость обеспечивается тем, что методика обучения быстро конфигурируется под новую задачу. Для настройки достаточно иметь физическую модель робота, которую можно дополнительно не подготавливать, а взять результат работы конструкторов. Также важно знать свойства используемых сервомоторов, чтобы измерить максимально допустимое усилие, которое может использовать мотор во время работы.

В ходе работы над проектом было замечено, что при повторном запуске процедуры обучения возможно получение различных моделей ходьбы. Это связано с тем, что пространство оптимизируемых параметров не является выпуклым и алгоритм обучения каждый раз может сходиться в различные локальные минимумы. Это означает, что имеет смысл проводить несколько процедур обучения и из них выбирать модель уже при помощи поиска наиболее качественной, где качество определяется при помощи человеческой разметки.

Хотя походка в симуляторе выглядит более эффективной, чем базовая, из-за того, что стопа не параллельна поверхности земли, походка кажется нереалистичной, так как в реальности робот может таким образом зацепиться за неровную поверхность. Поэтому для применения полученных результатов к реальному роботу, необходимо провести дополнительную работу.