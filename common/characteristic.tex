%% Рассматриваемый в дипломной работе робот состоят из неподвижных частей и сервомоторов, которые обеспечивают вращение вдоль определённой оси в заданном диапазоне градусов. 

Задача прямохождения двуногих роботов уникальна не только тем, что позволяет перемещаться по рельефной поверхности и преодолевать препятствия, но ещё и тем, что качество перемещения зависит от множества факторов окружающей среды и конструкции робота. Движение обеспечивается изменением положения подвижных деталей с заданной частотой. Алгоритм ходьбы генерирует последовательность движения сервоприводов, которая может быть постоянной, а может изменяться в ходе исполнения алгоритма, реагируя на меняющиеся условия.

Важно уметь решать эту задачу для произвольного робота. Так как в отличии от коммерческих моделей, которые поставляются с готовым функционалом, самодельные роботы не поддерживают доступные алгоритмы ввиду особенностей отдельных деталей.

Одним из первых решений является подход полного расчёта последовательности углов сервомоторов, основанный на физической модели робота, а также представлении походки в виде циклического изменения положения стоп робота \cite{inverse_1995}. Чтобы исправить проблему с низкой адаптивностью, в физическую модель была добавлена точка центра масс \cite{zmp_1999}. Были также подходы, в которых в качестве основной модели движения выбирается походка человека, а не спроектированная последовательность \cite{human_inspired}. Но все эти подходы обладают одним большим недостатком: они опираются на расчёты для заданной физической модели робота и предполагают, что ходьба совершается по ровной поверхности.

Вместе с математическим подходом развивается альтернативный подход, использующий методы машинного обучения. Один из первых алгоритмов опирался на конструкцию робота и оптимизировал последовательность стоп \cite{stochastic_2004}
. Этот подход позволял находить более оптимальные паттерны движения, но проблемы математического подхода оставались открытыми. В \cite{ml_in_bipedal_2012} были представлены подходы к обучению агента, где в качестве пространства состояний были выбраны углы поворотов моторов. Это давало большую свободу в выборе подходящей стратегии движения. Но алгоритм всё равно был ограничен, так как использовал дискретизированную версию действий агента – допускались повороты углов лишь на заданной конечное множество углов

В текущей работе будет изучена возможность применения методов обучения с подкреплением для задачи прямохождения самодельного робота. Задача будет решаться в симуляторе с упрощённой моделью, которая сохранила физические особенности реального робота. При обучении будет использоваться новейший метод, который обеспечивает быструю сходимость модели к оптимальному поведению \cite{kuznetsov2020controlling}.

Данный подход отличается от прочих тем, что исследуется возможность применения методов машинного обучения не на готовых настроенных средах, а на созданной среде, в которой при помощи экспериментов были найдены оптимальный набор используемых параметров робота, а также функция награды, которая приводит к обучению реалистичной походки, которую в дальнейшем можно использовать для ходьбы физической модели. Функция награды не зависит от физических свойств модели, поэтому полученный результат можно воспроизвести на других самодельных роботах.